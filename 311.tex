\documentclass[b5paper]{jlreq}  % 文章クラスとオプション設定

\usepackage[top=1cm,bottom=.8cm,left=1cm,right=1cm,foot=.5cm]{geometry}%用紙設定
\pagestyle{empty}%ページを出力しない%
\pagestyle{empty}%ページを出力しない%
\usepackage{luatexja}
\usepackage{tcolorbox}
\usepackage{ulem}%下線,波線,取り消し線など
\usepackage{tabularx}%伸びる表
\usepackage{emgyouretu}
%\usepackage{emathMw}
\usepackage{amsmath,amssymb,amsfonts}
%\usepackage{tikz-3dplot}
%\tdplotsetmaincoords{70}{115}
\usepackage{graphicx,color}% 
\usepackage{emath}
\usepackage{emathPs}% 
\usepackage{emathT}
%\usepackage[hiragino-pro]{luatexja-preset}
\usepackage[bizud]{luatexja-preset}
%\usepackage[shortlabels,inline]{enumitem}

%%箇条書き環境
% \setlist{noitemsep}
% \setlist[enumerate]{leftmargin=*}
% \setlist[enumerate,1]{label=\large\textbf{\arabic*.\hspace{.5cm}}, ref=rabic*}
% \setlist[enumerate,2]{label={(\arabic*)\hspace{.5cm}}, ref={(\arabic*)}}
% \setlist[enumerate,3]{label={\roman*)\hspace{.5cm}}, ref={\roman*)}}

\def\labelenumi{\Large\textbf\theenumi. }
\def\theenumii{\arabic{enumii}}
\def\labelenumii{(\theenumii)}
\def\theenumiii{\roman{enumiii}}
\def\labelenumiii{(\theenumiii )}%問題番号

\usepackage{hako}%ハコ環境
\usepackage{multicol}%段落環境
\setlength{\columnsep}{1cm}%段落間隔
%\setlength{\columnseprule}{0.4pt}
%\usepackage[continue]{emathAe}%解答環境
%\usepackage{type1cm}% PS, PDF 作成には必要
%\usepackage[debug]{emathPs}%描画eps作成
\usepackage{showexample}%\itemboxなど

% \def\thesection{No. \arabic{section}}
% \def\Name#1{\section{\large\bf #1\hfill\underline{ \hspace{1zw}年\hspace{2zw}組\hspace{2zw}番名前\hspace{12zw}}}}

%%%%%太字%%%%
%\def\bf{\gtfamily\bfseries}
\usepackage[e]{esvect} % パッケージをインクルード
\newcommand{\Beku}[1]{{\vv{\mathstrut #1 }}}
\renewcommand{\bekutoru}[1]{{\vv{\mathrm{#1}}}}
%問題・解答・コメントの表示・非表示

%\newif\ifquestion\questionfalse
\newif\ifquestion\questiontrue

\newif\ifanswer\answerfalse
%\newif\ifanswer\answertrue

%\newif\ifcomment\commentfalse
\newif\ifcomment\commenttrue

\def\thesection{\arabic{section}}
\def\thesubsection{}

\begin{document}

\begin{tcolorbox}[title= 現代数理統計学の基礎 第2章 問12 ,fonttitle=\gtfamily\sffamily\bfseries,colframe=black!50,colback=black!3]
    $X$の確率密度関数が$f(x)=1$,$0<x<1$,で与えられている。

    \begin{enumerate}[(1)\hspace{5mm}]
        \item $X$の積率母関数を求め,平均と分散を与えよ。
        \item $Y=X^2$なる変数変換したときの$Y$の確率密度関数を求め,その平均と分散を計算せよ。
        \item $Y=-\log(X)$なる変数変換したときの$Y$の確率密度関数を求め,その平均と分散を計算せよ。
        \item $\sigma>0$に対して$Y=\sigma X+\mu$なる変数変換をするとき,$Y$の確率密度関数,積率母関数,平均と分散を計算せよ。
    \end{enumerate}
\end{tcolorbox}

\begin{enumerate}[(1)\hspace{5mm}]
    \item $\begin{aligned}
        \bm{M_X(t)}&=E\left[e^{tx}\right]
        =\dint01e^{tx}f(x)dx
        =\dint01e^{tx}dx
        =\teisekibun{\dfrac{e^{tx}}{t}}01
        =\bm{\dfrac{e^t-1}t}\end{aligned}$\\
    
    
        
    
    
    $\begin{aligned}
        \bunsuu{d}{dt}M_X(t)&
        =\dfrac{e^t(t-1)+1}{t^2}\\
        \bm{E[X]}&=\lim_{t\to0}\dfrac{d}{dt}M_X(t)
        =\lim_{t\to0}\dfrac{e^t(t-1)+1}{t^2}
        =\lim_{t\to0}\dfrac{\teisei{t}e^t}{2\teisei{t}}
        =\bm{\bunsuu12}
    \end{aligned}$\\
        
    
    
    
    
    $\begin{aligned}
        \bunsuu{d^2}{dt^2}M_X(t)&
        =\dfrac{e^t(t^2-2t+2)-2}{t^3}\\
        E[X^2]&=\lim_{t\to0}\dfrac{d^2}{dt^2}M_X(t)
        =\lim_{t\to0}\dfrac{e^t(t^2-2t+2)-2}{t^3}
        =\lim_{t\to0}\dfrac{\teisei{t^2}e^t}{3\teisei{t^2}}
        =\bunsuu13\\
        \bm{\textbf{Var}(X)}&=E[X^2]-\left(E[X]\right)^2
        =\bunsuu13-\bunsuu14=\bm{\bunsuu1{12}}
    \end{aligned}$\\


    \item $y=g(x)=x^2\quad(0<x<1)$とすると,$g^{-1}(y)=\sqrt y$,$x:0\to 1$のとき,$y:0\to 1$\\
    
    $f_X(x)=f(x)=1$として,\\

    $\begin{aligned}
        \bm{f_Y(y)}&=f_X\left(\sqrt y\right)\cdot\zettaiti{\left(\sqrt y\right)'}
        =1\cdot\bunsuu1{2\sqrt y}=\bm{\bunsuu1{2\sqrt y}}\end{aligned}$\\
    
    
    $\begin{aligned}
        \bm{E[Y]}&=E\left[X^2\right]
        =\bm{\bunsuu13}\end{aligned}$\\



    $\begin{aligned}
        \text{Var}(Y)&=E\left[Y^2\right]-\left(E[Y]\right)^2=E\left[X^4\right]-\left(E[X^2]\right)^2
    \end{aligned}$\\



    
    $\begin{aligned}
        \bunsuu{d^3}{dt^3}M_X(t)&
        =\dfrac{e^t(t^3-3t^2+6t-6)+6}{t^4}\\
        \bunsuu{d^4}{dt^4}M_X(t)&
        =\dfrac{e^t(t^4-4t^3+12t^2-24t+24)-24}{t^5}\\
        E[X^4]&=\lim_{t\to0}\dfrac{d^4}{dt^4}M_X(t)
        =\lim_{t\to0}\dfrac{e^t(t^4-4t^3+12t^2-24t+24)-24}{t^5}
        =\lim_{t\to0}\dfrac{\teisei{t^4}e^t}{5\teisei{t^4}}
        =\bunsuu15\\
        \bm{\textbf{Var}(Y)}&=\bunsuu15-\bunsuu19=\bm{\bunsuu4{45}}
    \end{aligned}$


\newpage

    \item $y=g(x)=-\log x\quad(0<x<1)$とすると,$g^{-1}(y)=e^{-y}$\\
    
    $x:0\to 1$のとき,$y:\infty\to 0$\\
    
    $f_X(x)=f(x)=1$として,\\

    $\begin{aligned}
        \bm{f_Y(y)}&=f_X\left(e^{-y}\right)\cdot\zettaiti{\left(e^{-y}\right)'}
        =1\cdot e^{-y}=\bm{e^{-y}}\end{aligned}$\\
    
    
    $\begin{aligned}
        M_Y(t)&
        =\dint0\infty e^{ty}e^{-y}dy
        =\dint0\infty e^{(t-1)y}dy
        =\teisekibun{\dfrac{e^{(t-1)y}}{t-1}}0\infty
        =\bunsuu1{1-t}\quad (t<1)\end{aligned}$\\
    
     
    
    $\begin{aligned}
        \bm{E[Y]}&=\bunsuu{d}{dt}M_Y(t)\bigg|_{t=0}=\bunsuu{d}{dt}(1-t)^{-1}\bigg|_{t=0}
        =(1-t)^{-2}\bigg|_{t=0}=\bm{1}\\
        E[Y^2]&=\bunsuu{d^2}{dt^2}M_Y(t)\bigg|_{t=0}=\bunsuu{d}{dt}(1-t)^{-2}\bigg|_{t=0}
        =2(1-t)^{-3}\bigg|_{t=0}=2
    \end{aligned}$\\


    $\begin{aligned}
        \bm{\textbf{Var}(Y)}&=E\left[Y^2\right]-\left(E[Y]\right)^2=2-1^2=\bm{1}
    \end{aligned}$\\





\item $y=g(x)=\sigma x+\mu\quad(0<x<1)$とすると,$g^{-1}(y)=\bunsuu{y-\mu}{\sigma}$\\

$x:0\to 1$のとき,$y:\mu\to \mu+\sigma$\\

$f_X(x)=f(x)=1$として,\\

$\begin{aligned}
    \bm{f_Y(y)}&=f_X\left(\bunsuu{y-\mu}{\sigma}\right)\cdot\zettaiti{\left(\bunsuu{y-\mu}{\sigma}\right)'}
    =1\cdot \bunsuu1\sigma=\bm{\bunsuu1\sigma}\end{aligned}$\\


$\begin{aligned}
    \bm{M_Y(t)}&
    =\dint\mu{\mu+\sigma} e^{ty}\cdot\bunsuu1\sigma dy
    =\teisekibun{\bunsuu{e^{ty}}{t\sigma}}\mu{\mu+\sigma}
    =\bm{\bunsuu{e^{t(\mu+\sigma)}-e^{t\mu}}{t\sigma}}\end{aligned}$\\

    

$\begin{aligned}
    \bm{E[Y]}&=E[\sigma X+\mu]
    =\sigma E[X]+\mu=\bm{\bunsuu{\sigma}2+\mu}
\end{aligned}$\\


$\begin{aligned}
    \bm{\textbf{Var}(Y)}&=\text{Var}(\sigma X+\mu)
    =\sigma^2\text{Var}(X)=\bm{\bunsuu{\sigma^2}{12}}
\end{aligned}$\\



\end{enumerate}

    

            
\end{document}